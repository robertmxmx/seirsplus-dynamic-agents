\documentclass{article}
\usepackage[utf8]{inputenc}
\usepackage{graphicx}
\usepackage{cite}
\graphicspath{ {./images/} }



\title{Agent-based models of non-pharmaceutical interventions for epidemic control}
\author{Robert Brian Milligan and Supervised by Julian Garcia Gallego \& Buser Say}
\date{7 July 2022}


\begin{document}


\maketitle

\begin{abstract}
	Currently there is much interest in modelling diseases to understanding ways governments, workplaces and other decisions makers may seek to control the spread of COVID-19. Many various computational models of disease spread exist and this paper builds upon an network agent based model and changes the ways in which agents can comply or not compy with various actions. Both non-stategic and strategic methods of agent compliance were looked into and it was found that the lower the reproduction rate of the disease the better any type of non-pharaceutical intervention was, with high R0 values the best results were found when benefit for compliance related to the number of agents one was in contact with, and with lower R0 values the best results were found when the benefit for compliance related to looking at the number of agents one was in contact with that was either symtomatic, hospitalised, fatality or in quarantine.
\end{abstract}



\tableofcontents

\newpage 

\section{Introduction}

Mathematical and computational models are important in preparing policies to deal with pandemics. Mathemeaickl
These models typically do not incorporate behaviour or if they do, do so in simple ways such as compartmental models that incorporate "aggregate states" this was implemented as having 6 diiferent behavioural secenarios which would change based on the time of the simulation or having a certain threshold of positive test, positive cases or deaths per day ~\cite{karaivanov_2020}

This paper investigates the modification of an existing model diease spread produced by Ryan McGee and used a part of various peer reviewed research articles ~\cite{mcgee_homburger_williams_bergstrom_zhou_2021} ~\cite{mcgee_homburger_williams_bergstrom_zhou_2021_2}. 
I have looked at applying the model to the situation of COVID-19 in childcare and involves creating both a non stategic and strategic behavoioural models. 
The Behaviour relates to compliance to various actions the agent can choose to do such as doing an additional COVID rapid antigent test if they have symtoms of the virus. The non strategic model looks at giving agents a fixed cost of complying with requested actions with a reward based on a mix of local situations such as if they are in close contact with an agent who is in isolation and a global situation such as the percentage of the group that reported a positivie test within the last 2 weeks. The strategic model looks at agents where they know how all other agents will act and use this information to decide if it in their interest to comply.

Add a few more paraggraphs: go slowly.


The rest of this paper is organised as follows. Section \ref{description} describes the basic model, Section....



\section{Description of Model \label{description}}
The model that is modified in this paper is an Agent Based Model (ABM). This ABM is one where agents can belong to one and only one compartment representing their state. 

Agents progress though states with the only sink node being Recovered and Fatality
the states of Susceptible, Exposed, Pre-Infectious, Infectious-Asymptomatic, Infectious-Symptomatic, Hospitalised, Fatality and Recovered exist. All these except Hospitalised and Fatality can have agents in a mirror state where they are also isolated, meaning they cannot acquire the disease or spread it to any other agents in the network.

Once an agent catches the disease they move along the stages of the disease until they are a fatality or recovered, at any point they have the ability to move across to the mirror isolation state. 





\begin{figure}
  \centering
      \includegraphics[width=\textwidth]{SIR}
  \caption{a diagram of the existing model of seirplus ~\cite{mcgee_2021}}
\end{figure}

\newpage



\subsection{Using networks to model contacts}

The version of the model we use considers a simple network, one which uses a one single connected network designed to simulate a workplace.

These large networks are made up of a number of cohorts which are loosely connected and each of these cohorts can have a number of subgroups which are highly connected.

The network is set up before the simulation begins and is different every time and does not change throughout the simulations run. 

Agents are arranged in a network as nodes and are connected by edges representing which other nodes are their close contacts. 80\% of disease spread is though the close contact edges and 20\% is spread randomly.

Each day agents will be asked to do a test if their day has come up on a surveillance testing schedule, they show symptoms or have been contacted that they are a close contact. However there is a compliance value to them following through with the requested action

During the day agents can spread the contagion to each other and can progress though the stages of the disease if they have it. They additionally have the choice to participate in contact tracing.

Agents will also be asked to isolate for one of six reasons. They or a group member develop a symptomatic case, returns a positive test or is told they are a close contact though using contact tracing. Again there is a compliance value to decide if they follow through with the requested action

These systems can effectively be disabled by overriding the compliance for them to be a large negative number, for example compliance with contact tracing can be set to -10 to disable the use of that system and this done to limit the model to a smaller number of variables which compliance can affect.


The model has a variety of limitations including 
\begin{itemize}
\item Having all agents always test on the same day, thus if semiweekly is chosen, all agents will be asked to test on Monday and Thursday
\item the network itself is unchanging, however this is not necessarily bad as it removes a variable that may change results
\end{itemize}

\newpage

\begin{figure}
  \centering
      \includegraphics[width=\textwidth]{network}
  \caption{An Example Netowrk}
\end{figure}

Example of the network of 100 agents
To simulate the childcare scenario they are split into 5 groups of about 20 agents each with high connectivity inside the group and low connectivity between the groups 
The agents current state is shown by the nodes colour and the edges are that agents close contacts which the disease can spread easiest though the population


\subsection{Description of Model Parameters}
The Current Modifications of the model relate to allowing 10 of the Model Parameters that relate to Compliance to be dynamically updated each day dependent on a given rule. The 5 most relevent of these being.
\begin{itemize}
\item What proportion of agents take a test immediately as a result of having symptoms
\item What proportion of agents will do surveillance
\item What proportion of agents will isolate given they have a symptomatic case
\item What proportion of agents will isolate given a positive result from a test
\item What proportion of agents in a group isolate given one of them has a positive result from a test
\end{itemize}

A simple model might only use a few of these compliance parameters such as  and is how the parameters work in the base model

\begin{table}[h!]
\begin{tabular}{ll}
Symtomatic Testing Rate Compliance & 50\% \\
Surveillance Testing Rate Compliance & 50\% \\
Isolation from Positive Test Compliance & 100\%
\end{tabular}
\end{table}


\subsection{Description of Compliance In the Model}
Compliance can be set a variety of ways in the modified model but in this paper 3 are used
\begin{itemize}
\item The default setting where agents are given an inital value for compliance and is unchanging e.g. 50\% are set to comply and will always do so
\item The non stategic model where agents are given an inital compliance and can become more compliant depending on the network situation , their local situation or a mix
\item The stategic model where agents utilise knowledge of the amount of agents in the network who will comply to make a decision to comply or not
\end{itemize}



\section{Childcare Model and Benchmark Tests}

\begin{itemize}

\item Parameter justification
\item test false negative rate 0.36 ~\cite{van_de_mortel_2022}
\item R0 mean of Omicron is 9.5, delta is 5.4 ~\cite{liu_rocklov_2022}
\item base compliance for behaviour is set at levels of  0.5 and 1, to represent 50\% of agents and all agents complying
\item 100 agents in model split into 5 groups with high interconnectedness within the group and low connections between groups

\item compliance increases for 2 actions,  symtomatic test and regular interva survillencel tests
\item The cost of compliance is is fixed and there is a reward based on if connecting agents are isolated, hospitalised or a fatality

\item from looking at early results since only 2-3 tests a week there is a delay and the higher the r0 the fewer chances they can have to change their mind about compliance 

\item the R0 of 9.5 or 5.4 may be too high as it assumes children spread the disease at the same rate as the general population as well as all disease spread occuring in childcare which it probably is not. Therefore 2 other R0 cases of 3 and 2 are used as they may more accuratley capture a real world scenario where not all transmision is though the 1 childcare setting
\end{itemize}

\begin{figure}
  \centering
      \includegraphics[width=\textwidth]{Figure3}
      \includegraphics[width=\textwidth]{Figure3Net}
  \caption{Benchmark Run with Unchanging Compliance at 50\% for regular twice weekly survilence testing and testing if symtomatic with the resulting final network. 92 of the 100 agents received the infection}
\end{figure}

\begin{figure}
  \centering
      \includegraphics[width =150pt]{basicnet}
  \caption{An Example Simplified Network}
\end{figure}

The benchmark model is associated with one single large peak of disease spread

\newpage

\section{Non-Strategic Model}
We have a fixed cost of compliance to an action and varying factors that can raise it. A mix of global and local behavioural factors can be added to make an agent more compliant. In the current build these are the known positive cases in the network in the past 14 days. The proportion of contact agents (those which share an edge) which are a fatality, hospitalised or in isolation. These values are taken away from the base cost and if the result is lower from a specified value the agent will be compliant to that action, otherwise they are not

For a simple example we have a situation for this central agent 


\begin{table}[h!]
\begin{tabular}{ll}
Symtomatic Testing Rate Compliance & 50\% \\
Surveillance Testing Rate Compliance & 50\% \\
This agent’s base aptitude & uniformly distributed (-0.1,0.1) \\
Base Cost of Compliance & 0.5 
\end{tabular}
\end{table}

Compliance = 0.5 – (5 * 1/10) – (4*0.02) + 0.3 = 0.22



In this case the agent is compliant
it will test if they develop a symptomatic case immediately and will do surveillance testing as the compliance is now less than the threshold of 0.5
if the neighbouring agent that entered hospital returns a positive test, our agent will enter isolation if a positive test is returned 
the reward for agents can be based on a mix of both a local and global network situation as in the example runs

Of the 10 possible compliance parameters built into the existing mode, 3 are used these which are:
\begin{itemize}
\item immediately testing if the agent has a symptomatic case
\item using regular surveillance testing
\item isolating if a positive test is returned - set to always be true
\end{itemize}

\section{Strategic Model}
The view of benefit and cost can be grouped into 3 categories.
There is a level at which people will not contribute as they find the act pointless as they know near no agents  will comply 
\begin{itemize}
\item1. Global Situation of the model e.g. changes over time but is idential across all agents
e.g. number of active cases or rate of change in spread rate , could place 2 infections in first week and compare each week to the last week, rate of change week on week
[0,0,0,0,0,0,2]  if the next week 4 cases are spread, the rate would be 2
\item2. Structural Layout of the model, identical over time but varies between each agents
e.g. number of close contacts or number of agents one can reach in 1 step , or could be 2 steps, the more agents one is in contact with the greater the benefit 
\item3. Structual Situation of the model, changes over time and varies for each agent
e.g. number of close contacts in a particular state or group of states like hospitalised, fatality, isolated etc, if a close contact is in one of those states the benefit to compliance is greater
\end{itemize}


\section{Results}

In this model two choices of compliance are considered, If an agent is compliant they will perform semi-weekly surveillance testing in addition to testing whenever they develop a symptomatic case of the disease, If they are not compliant they will do neither of these actions.

There are 5 models compared these are
Baseline 50\% compliance, this is where each agent when generated has a 50\% chance to always or to never comply. 

Baseline 100\% compliance, this is where each agent when generated has a 100\% chance to always comply. 

for the non-strategic model, initially a value is generated for what threshold of compliance they will have and when it is past that set value they become compliant and if it moves back they no longer are. For example the base attitude for an agent might be 0.5, if on a given day the benefit exceeds that they become compliant and less than that, no longer compliant.

Non-Strategic Minimum 50\% , this is where each day the compliance value for an agent is updated and changes based on a combination of the global known positive test cases recorded in the network within the last 14 days as well as the proportion of close contacts of an agent that are either symptomatic, a fatality, hospitalised or in isolation.

For the strategic model a curve is created to model a game theory dilemma on wherever or not the agent should comply given they know the reward for compliance and what every other agent will do, depending on the benefit the likelihood of any one agent complying can vary from 0, to 60-99% to 100%.

Strategic Community Size, this is where the benefit to compliance is based on the number of close contacts the agent has. This is a case where the curve is different for each agent but unchanging over time.
Strategic Local State, this is where the benefit to compliance is based on the number of close contacts that are either in a state of symptomatic, a fatality, hospitalised or in isolation. This is a case where the curve is different for each agent and unchanging over time.
Strategic Global State, this is where the benefit to compliance is based on the number of agents in the network that are either in a state of symptomatic, a fatality, hospitalised or in isolation. This is a case where the curve is the same for each agent but changes over time.


\begin{figure}[h!]
  \centering
      \includegraphics[width=\textwidth]{5}
  \caption{Test Results over R0 of 2}
\end{figure}

\newpage
\begin{figure}[h!]
  \centering
      \includegraphics[width=\textwidth]{4}
  \caption{Test Results over R0 of 3}
\end{figure}

\begin{figure}[h!]
  \centering
      \includegraphics[width=\textwidth]{3}
  \caption{Test Results over R0 of 5.4}
\end{figure}
\newpage
\begin{figure}[h!]
  \centering
      \includegraphics[width=\textwidth]{2}
  \caption{Test Results over R0 of 9.5}
\end{figure}

\begin{figure}[h!]
  \centering
      \includegraphics[width=\textwidth]{6}
  \caption{Test Results over all R0 values at 100\% Compliance}
\end{figure}


\begin{figure}[h!]
  \centering
      \includegraphics[width=\textwidth]{1}
  \caption{Test Results Comparing the Results of R0 values}
\end{figure}
\newpage

\section{Analysis of Testing Compliance}

To evaluate the result of the model we take tghe average of these two values over many runs 
\begin{itemize}
\item What percentage of the population caught the contagion?
\item How many days did it take for the outbreak to stop with 0 actual active cases?
\end{itemize}

We can analyse compliance by comparing the effect varying levels have on the length a contagion actively spreads and what proportion of the population becomes infected. Firstly a baseline can be set to explore the parameter space, then further tests done to see the effect having compliance change as a result of the current known spread of the contagion in the network.


Over 4 different R0 virus base reproduction rate was the model run with, over the different R0 levels the reducion in total cases reduced from a -2\% improvement with an R0 of 9.5 to a 28\% improvement with an R0 of 2. This is probably due to one of the aspects the model relating to the 14 day known positive cases recorded. With a low R0 in the system there are more chances for agents to test themselves before they pass on the disease and these reported cases incentivise even more agents to do symptomatic and surveillance testing. With extremely high R0 numbers it is observed that the non-strategic behavioural model has little effect on the total number infected. Across all cases it can be seen that the amount of time the simulation runs for is tied with how many agents get infected. Both the static values of compliance and non-strategic behavioural model follow this trend. 


\section{Discussion}

\section{Conclusion}

\newpage
\appendix

\section{Appendix Section}

Removed parameter section
\begin{itemize}

\item TRACING\_COMPLIANCE\_RATE (what proportion of agents comply with contact tracing)
\item ISOLATION\_COMPLIANCE\_RATE\_POSITIVE\_CONTACT (what proportion of agents isolate given they are a close contact) 

\item TESTING\_COMPLIANCE\_RATE\_TRACED (what proportion of agents take a test immediately as a result of being informed they are a close contact)


\item ISOLATION\_COMPLIANCE\_RATE\_SYMPTOMATIC\_GROUPMATE (what proportion of agents will isolate given one of their group mates has a symptomatic case)

\item ISOLATION\_COMPLIANCE\_RATE\_POSITIVE\_CONTACTGROUPMATE (what proportion of agents in a group isolate given one of them is a close contact)
\end{itemize}

\bibliography{bibliography}{}
\bibliographystyle{plain}

\end{document}


\end{document}
