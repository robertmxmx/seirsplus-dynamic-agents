\documentclass{article}
\usepackage[utf8]{inputenc}
\usepackage{graphicx}
\usepackage{hyperref}
\usepackage{cite}
\usepackage[margin=1.35in]{geometry}
\graphicspath{ {./images/} }

\title{The limits of voluntary testing: strategic compliance in SIR models}
\author{Robert Brian Milligan,  Buser Say \& Julian Garcia}
\date{}


\begin{document}


\maketitle

\begin{abstract}
Epidemic modelling has proven vital for understanding ways in which governments, workplaces and other decisions makers seek to control the spread of COVID-19. Although many computational models of disease spread exist, few consider human behaviour explicitly. In this paper, we extend an agent based network model to account for ways in which agents can comply with voluntary non-pharmaceutical interventions. We investigate non-strategic and strategic models of agent compliance and find that the lower the natural reproduction rate of the disease, the higher the chance of achieving success with voluntary non-pharmaceutical interventions. For high R0 values, the best results found were when the choice of compliance is based on the number of close contacts an individual has. Whereas for lower R0 values, the best found results were when choice of compliance related to agents considering the number of higher risk close contacts they have. Higher risk in his case referred to agents who were symptomatic, hospitalised, fatality or in quarantine. In short, voluntary non-pharmaceutical interventions work best when R0 is relatively low.
\end{abstract}




\newpage 

\tableofcontents

\newpage 

\section{Introduction}

Mathematical and computational models are important in designing policies to deal with epidemics. 
Standard mathematical models typically assume an infinite population and place proportions of this population into compartments ~\cite{cooper_mondal_antonopoulos_2020}. 
The traditional compartmental model is known as SIR, ({\bf S}usceptible, {\bf I}nfectious and {\bf R}ecovered). 
Extensions to this basic model typically add new compartments and flows between them to account for additional elements or assumptions. 
Some common examples of variations on SIR models are SEIR, SIS and SIRV. 
SEIR, for example adds a new Exposed compartment to account for incubation periods. 
SIS models assume that once an individual recovers from the disease, the recovered individual is not immune and re-enter the susceptible compartment.
SIRV models add a Vaccinated state whereby susceptible individuals who move to the vaccinated compartment cannot become infected with the disease.
When solving these models mathematically, numerical solutions can be found with standard numerical methods. \newline

When agent heterogeneity is important, as it is the case in this paper, agent based models are a popular choice. 
These models typically have a finite number of agents which interact with each other in some manner and can add greater granularity to the model.
Moreover, these models can allow for agents to be heterogeneous with their personal attributes.\newline

Most epidemiological models do not incorporate behavioural elements, or if they do, they do it in simple ways such as compartmental models that incorporate "aggregate states". One such example is Karaivanov 2020 ~\cite{karaivanov_2020}. This paper considers six different behavioural scenarios that would change based on the time of the simulation or having a certain threshold of positive test, positive cases or deaths per day.\newline

This paper investigates the extension of an existing model disease spread produced by Ryan McGee and used as a part of various peer reviewed research articles ~\cite{mcgee_homburger_williams_bergstrom_zhou_2021,mcgee_homburger_williams_bergstrom_zhou_2021_2}. 
We have applied this model to simulate COVID-19 in a childcare setting and extended this model with the addition that the compliance of agents is controlled with non strategic and strategic behavioural choice models. Specifically, we model the behaviour of agents in terms of their compliance to the voluntary surveillance testing and additional testing if an agent shows symptoms. \newline

The non strategic model gives agents a fixed cost of compliance to perform requested actions with each individual calculating a perceived reward. A mix of local conditions determine the reward such as if the agent has a close contact who has a symptomatic case of the disease, and a global situation such as the percentage of the group that have reported a positive test within the last two weeks. The strategic model considers agents which know how all other agents will act and use this information to help them decide if it in their interest to comply. This incorporates a cost of compliance, a reward as well as a minimum number of agents who need to be compliant to see a reward for the group. Both these models are compared to a baseline model and ran over a series of four different R0 values (The disease basic replication rate). \newline

The rest of this paper is organised as follows. Section 2 describes the base model. Section 3 describes the base benchmark model. Section 4 introduces the non-strategic model. Section 5 introduces the various strategic models. Section 6 shows the results of running the models. Finally, section 7 discusses the results and future work.



\newpage
\section{Description of the Base Model \label{description}}

In this section we describe the base model \footnote{SEIRS+ github repository {\url{https://github.com/ryansmcgee/seirsplus}}}, for which extensions are presented  \footnote{Our extended version of SEIRS+  {\url{https://github.com/robertmxmx/seirsplus-dynamic-agents}}}. 
This model is an Agent Based Model (ABM). 
The model was specifically designed to evaluate the effects of surveillance testing and partial vaccination for COVID-19.\newline

In an ABM, Agents can belong to one and only one compartment representing their state. 
When an agent becomes infected they progress though states ending up either in the Recovered or Fatality. 
The model considers the states of Susceptible, Exposed, Pre-Infectious, Infectious-Asymptomatic, Infectious-Symptomatic, Hospitalised, Fatality and Recovered as seen in figure 1.
Susceptible agents represents agents who have not yet come in to contact with the disease. 
Exposed agents represent are those agents who have interacted with an infectious agent and have caught the disease.
Agents then move to Infectious Pre-Symptomatic. In this state they can spread the disease, have no symptoms and can show up as positive if the agent takes a test. The addition was added so that agents who will develop a symptomatic case will not test before they develop symptoms but can still spread the disease.
Agents then move to either Infectious Asymptomatic or Infectious Symptomatic, the difference being is that asymptomatic cases are unaware they have the disease. Symptomatic cases may take a disease test as a result of them noticing symptoms and have the chance of moving into the hospitalised group. These two compartments were added as COVID-19 can be experienced as a Symptomatic or Asymptomatic disease. Agents who are Infectious Symptomatic can progress into Hospitalised and then potentially Fatality. Agents in either of these groups cannot spread the disease to anyone else in the network.
Recovered agents represent those once infectious agents who have fully recovered and gained immunity to the disease. 
Importantly the model assumes that once an agent recovers from the disease they can no longer become infectious. \newline 
All the compartments except Hospitalised and Fatality can have agents in a mirror state where they are also isolated, meaning they cannot acquire the disease or spread it to any other agents in the network. Once an agent catches the disease they move along the stages of the disease until they are a fatality or recovered, at any point they have the ability to move across to the mirror isolation state. Agents who are in a Hospitalised, Fatality or in a Quarantined state cannot pass on the disease to anyone else in the network.


\begin{figure}[h!]
\centering
\includegraphics[width=\textwidth]{SIR}
\caption{A diagram of the existing SEIRS+ model with its various compartments ~\cite{mcgee_2021}.The dashed arrows show those agents can move back and forth between. However. for the implemented models in this paper isolation compliance is set at 100\%, thus agents who have entered Quarantine will only ever leave Infectious Asymptomatic (Quarantine) to go to Hospitalised, or from Recovered (Quarantine) to Recovered.}
\end{figure}

\newpage


\subsection{Network Based Model}

We use the version of SEIRS+ that uses one simple network connected network designed to simulate a workplace. These large networks are made up of a number of cohorts that are loosely connected and each of these cohorts can have a number of subgroups that are highly connected. The network is set up before the simulation begins and is randomly generated every time the simulation is run. Once the network is generated its structure does not change throughout the simulations run.  Agents are represented in this network by nodes pair of close contacts are represented by an edge. 80\% of disease spread is through the close contact edges and 20\% is spread randomly. These close contacts are the way in which we assign the agents, view of their local setting.


\newpage


\subsection{Compliance}
The model uses a variety of parameters some of which relate to the compliance of agents to perform certain actions when asked. For this paper those compliance parameters which are related to surveillance testing, testing if showing symptoms and quarantining if receiving back a positive test are considered. Each day agents will be asked to do a disease test if their day has come up on a surveillance testing schedule or if they currently have a symptomatic case of the disease. However, there is a compliance value to decide if they go through with the requested action. The base model uses more than these three compliance parameters, but they are effectively removed by setting their compliance value to a negative number so that they never occur. In the existing model a parameter is set that assigns a proportion of agents to always comply and the remaining to never comply. This is set at the start of the simulation and never changes. In this expansion of the SEIRS+ the model is changed so that this value is recalculated on every day of the simulation. This is done by setting each agent with an initial compliance value in a given range. If the benefit to compliance on that given day is greater than that initial compliance value then they become compliant. \newline

This leads to three systems of compliance which will be discussed in detail in the latter sections. The default setting where agents are given an initial value for compliance and is unchanging e.g. 50\% are set to comply and they will always do so. The non strategic model where agents are given an initial compliance and can comply with a higher probability depending on the agents view of those other agents around them and network statistics they may have access to. The strategic model where agents utilise knowledge of the number of agents in the network who will comply to make a decision to comply or not.\newline

It should be noted the model has a few limitations. Having all agents always test on the same day, thus if semi-weekly is chosen, all agents will be asked to test on Monday and Thursday each week. The structure of the network is fixed and has not ability for agents to join, leave or modify who their close contacts are. Additionally, all values for the false negative rate for tests, fatality and hospitalisation rates are the same for all agents. This is a result of the existing model not having a system where agents can be assigned different ages in the same network and the ability to assign different false negative rate for individual or types of agents.





\newpage
\section{Benchmark Baseline Model}

We choose to create our benchmarks around COVID spread in childcare settings. 
This is an interesting case, because we focus on voluntary surveillance testing. This is how the Victorian government chose to do surveillance in childcare scenarios. They provide free antigen tests, allowing the community to decide whether to test or not \footnote{See more at \url{https://www.premier.vic.gov.au/supporting-families-free-rats-school}}.

In addition, this allows us to consider networks of relatively small size, which implies we do not need massive computational infrastructure to produce results. \newline 


The network structure of the model can be used to reflect the dynamics that would be apparent in a child care building and parameters could be chosen to match this specific scenario.\newline

\begin{tabular}{|c|c|}
\hline
\textbf{Parameter} & \textbf{Value} \\ \hline
Test False Negative Rate (RAT) & $0.36$ as per~\cite{van_de_mortel_2022} \\ \hline
R0 &  $9.5$ or $5.4$ as per Omicron and  \\
&  Delta variants respectively~\cite{liu_rocklov_2022} \\ \hline
Number of Survillence Tests A Week&  Two each week ~\cite{premier_of_victoria_2022} \\ \hline
Number of Agents & $100$\% \\ \hline
Number of Cohorts & $5$, typical number of rooms in  \\ 
& a childcare setting \\\hline
Mean Number of Connections Between Cohorts  & $6$ \\ \hline
Density of Edges in Cohort Networks  & $0.15$ \\ \hline
\end{tabular}
\newline


\begin{figure}[h!]
\centering
\includegraphics[width=\textwidth]{network}
\caption{An example network of 100 agents. Five distinct clusters can be observed with high connectivity within the groups and low connectivity between the groups. It is also observed that three of the four exposed agents are those who are a close contact of an infectious agent}
\end{figure}

\begin{figure}
\centering
\includegraphics[width=\textwidth]{Figure3}
\includegraphics[width=\textwidth]{Figure3Net}
\caption{Benchmark Run with Unchanging Compliance at 50\% for regular twice weekly surveillance testing and testing if symptomatic with the resulting final network. 92 of the 100 agents received the infection}
\end{figure}

\newpage 

To simulate the childcare scenario the 100 agents are split into 5 groups of about 20 agents each with high connectivity inside the group and low connectivity between the groups. The edges in the network represent that particular agent's close contacts which the disease can spread easiest though. For the benchmark baseline, compliance for behaviour is set at levels of 0.5 and 1, to represent 50\% of agents and all agents complying. If agents are compliant, they will perform symptomatic test and regular interval surveillance tests when asked.\newline

The R0 of 9.5 or 5.4 may be too high as it assumes children spread the disease at the same rate as the general population as well as all disease spread of children occurring in entirely in the childcare location which is probably not the case. Therefore two other R0 cases of 3 and 2 are used as they may more accurately capture a real world scenario where not all transmission is though the childcare setting. Additionally statistics are not currently available for an accurate R0 for young children to other young children, therefore the general population spread rates had to be relied upon. \newline

Two different runs of the benchmark model are run. Firstly a baseline of 50\% compliance, where each agent when generated has a 50\% chance to always or to never comply. The other is a baseline 100\% compliance, where each agent when generated has a 100\% chance to comply. 


\section{A Non-Strategic Behavioural Model}
The non-strategic model has two main components, namely (i) the cost and (ii) benefit of compliance, where the benefits must outweigh the cost for an agent to be compliant.
The cost is fixed at a set value and given a value so that initially 50\% of agents will be compliant. 
A combination of global and local behavioural factors are be added to set an agents benefit, making an agent more compliant.
The global factor is the known positive cases in the network in the past 14 days. 
The local factor is the proportion of contact agents (those which share an edge) which are a fatality, hospitalised or in isolation. 
The benefit is taken away from the base cost and if the result is lower than a specified value, the agent will be compliant to that action, otherwise they are not.\newline

\begin{figure}[h!]
\centering
\includegraphics[width =150pt]{basicnet}
\caption{An example simplified network showing a single agent and their close contacts}
\end{figure}


As per figure 4 this would inclicate one of the eight close contacts are in a high risk category. With a case where 1\% of agents in the full network have reported a positive case in the last two weeks. The calculated compliance would be one where the agent is compliant as the result is under 0.5.

\[Compliance = 0.5 – (5 * 1/8) – (5*0.01) + 0.1 = -0.075\]


The agent would potentially become uncompliant if the close contact in isolation, recovered and left isolation as the new compliance equation would become one over 0.5 where the agent is no longer compliant. This can be see based on the calculation provided below, where the agent's compliance has gone above the 0.5 threshold.
\[Compliance = 0.5 – (0 * 1/8) – (5*0.01) + 0.1 = 0.55\]



Agents are given diversity in the non-strategic model. Initially a value is generated that is used in all compliance calculations for that agent. This either biases them to be more or less compliant, for example the base attitude for an agent might be 0.1; meaning the benefit would have to be 0.1 greater than that of a completely neutral agent for them to comply. This model is used for one case where the fixed cost is set at 0.5, this model is called the Non-Strategic Minimum 50\%.  In this model each day the compliance value for an agent is updated and changes. Factors that change the agents compliance a combination of the global known positive test cases recorded in the network within the last 14 days as well as the proportion of close contacts of an agent that are either symptomatic, a fatality, hospitalised or in isolation.\newline



\begin{tabular}{|c|c|}
\hline
Symptomatic Testing Rate Compliance & 50\% \\ \hline
Surveillance Testing Rate Compliance & 50\% \\ \hline
This agent’s base aptitude & 0.1, and is uniformly distributed from (-0.1,0.1) \\ \hline
Base Cost of Compliance & 0.5 \\ \hline
\end{tabular}
\newline


\newpage
\section{A Strategic Behavioural Model}
For the strategic model a strategic Stag Hunt Game is played similar to the one explored in this paper~\cite{pacheco_santos_souza_2014}. A curve is created to model a game theory dilemma on wherever or not each agent should comply, given they know the cost and reward for compliance and what every other agent will do. Depending on the benefit the agent sees for compliance, the likelihood of any one agent complying can vary from 0, to 60-99\% to 100\%. There is a level at which people will not contribute as they find the act pointless as they know near no agents will comply. This model is used in three cases which are discussed later and is designed like the non-strategic model to have a baseline of 50\% compliance. This will then grow based on the situation of each agent.\newline

\begin{tabular}{|c|c|}
\hline
Parameter & Value \\ \hline
Cost of Compliance & 1 \\ \hline
Overall benefit If action sucsessful & Depends on either a global or local situation\\ \hline
Minimum number of agents in & 60\\ 
group complying to see benefit & \\\hline
Number of agents in group & 100\\ \hline
\end{tabular}
\newline

The value benefit is assigned can be grouped into three categories. Strategic Community Size is where the benefit to compliance is based on the number of close contacts the agent has. This is a case where the curve is different for each agent but unchanging over time. Strategic Local State is where the benefit to compliance is based on the number of close contacts that are either in a state of symptomatic, a fatality, hospitalised or in isolation. This is a case where the curve is different for each agent and unchanging over time. Strategic Global State is where the benefit to compliance is based on the number of agents in the network that are either in a state of symptomatic, a fatality, hospitalised or in isolation. This is a case where the curve is the same for each agent but changes over time. \newline

\begin{tabular}{|c|c|c|}
\hline
Model & Benefit Curve & Benefit Curve\\ 
&is different for each agent &changes over time \\\hline
Strategic Community Size & Yes & No \\ \hline
Strategic Local State & Yes &Yes\\ \hline
Strategic Global State & No &Yes\\ \hline
\end{tabular}
\newline


\begin{figure}[h!]
\centering
\includegraphics[width=12cm]{strat}
\caption{Comparison of two extreme examples of compliance curves with a very low (50) and very high (1000) benefit to compliance, the game is run with a cost of 1, 100 agents and 60 must be compliant to see a benefit}
\end{figure}

In figure 5 the chart on the left has two points of compliance 0 and 0.65 . This is because if the agent was between 0 and 0.55 they would see compliance as not worth it as their overall benefit to compliance in the system is negative and move towards a compliance of 0. If the agent was between .55 and 0.65 they would move towards 0.65 as the benefit to compliance is still positive for more agents to comply. Finally if the agent was between 0.65 and 1 they would move towards 0.65 as the benefit to compliance is negative, so the additional compliance is not rewarded. This creates a system with two absorbing states 0 with a weight of 0.65 and 0.65 with a weight of 0.35, in this case the state of 0 is dominant, so this agent would comply 0\% of the time. \newline\newline
On the other hand, the chart on the right has two points of compliance at 0 and 1. If an agent was between 0 and 0.45 they would see compliance as not worth it as their overall benefit to compliance in the system is negative and move towards a compliance of 0. If the agent was between 0.45 and 1 , benefit is positive for more agents to comply. This creates a system also with two absorbing states, 0 with a weight of 0.45 and 1 with a weight of 0.55, in this case the state of 1 is dominant, thus the agent would comply 100\% of the time.


\section{Results}

In this section, we report the results of our computational experiments where we evaluate the effect of the three broad model types. We investigate the models effect on two key measures, the percentage of agents that contracted the disease and the number of days that elapsed until zero agents were infectious.  In this model two states of compliance are considered. If an agent is compliant they will perform semi-weekly surveillance testing in addition to testing whenever they develop a symptomatic case of the disease. If the agent is not compliant they will do neither of these actions. All agents will always isolate immediately after receiving a positive test result. The results of these models are computed from 50 separate runs for each, model and R0 combination. \newline

We compared six models are which consist of three broad types, these include the two baselines, one non-strategic and three strategic models. The baselines are where agents have a probability to be complaint initially and do not change from the initially set value. Baselines of 50\% and 100\% are used. The non-strategic minimum 50\% model makes agents compare the benefits of compliance with a fixed cost. Perceived benefit is a combination of the 14 day positive case rate and proportion of high agents who are close contacts. The three strategic models use a strategic Stag Hunt Game where agents consider the cost, benefit and the proportion of agents who need to be compliant for the benefit to occur. The strategic models differ with how the perceived benefit for each agent is calculation. Strategic community size considered how many close contacts an agent has, Strategic local state considers how many agent’s close contacts are in a higher risk state and Strategic global state considers how many agents in the entire network are in a higher risk state. A summery of the six models are shown in the table bellow. \newline

\begin{tabular}{|c|c|}
\hline
Model & Explaination \\ \hline
50\% baseline & half of agents are set to always comply \\ \hline
100\% baselinel & all agents are set to always comply \\ \hline
Non-Strategic Model & decision to comply is based on\\ 
 & a global known number of positive cases\\
 & and a local proportion of higher risk close contacts\\\hline
Stratgeic community size & number of close contracts\\\hline
Stratgeic local state & number of higher risk close contacts \\\hline
Stratgeic global state & number of higher risk agents in network\\ \hline
\end{tabular}

\newpage

\begin{figure}[h!]
\centering
\includegraphics[width=\textwidth]{5}
\caption{Test Results over R0 of 2. Strategic Local State had the greatest overall range in infected agents but also its mean was the lowest.}
\end{figure}

\begin{figure}[h!]
\centering
\includegraphics[width=\textwidth]{4}
\caption{Test Results over R0 of 3. The Non Strategic model achieved a lower infected rate than all of the strategic model types.}
\end{figure}

\newpage

\begin{figure}[h!]
\centering
\includegraphics[width=\textwidth]{3}
\caption{Test Results over R0 of 5.4 (Delta). The Strategic Local state performed significantly worse than all other models}
\end{figure}

\begin{figure}[h!]
\centering
\includegraphics[width=\textwidth]{2}
\caption{Test Results over R0 of 9.5 (Omicron). The infected rate for all models was near identiical and generally not effective at all in reducing disease spread. }
\end{figure}
\newpage

\begin{figure}[h!]
\centering
\includegraphics[width=\textwidth]{6}
\caption{Test Results over all R0 values at 100\% Compliance. Testing and isolation has little effect on stopping disease spread with everyone complying at a R0 of 9.5 .}
\end{figure}

\begin{figure}[h!]
\centering
\includegraphics[width=\textwidth]{1}
\caption{Test Results Comparing the Results of R0 values. There is an overall trend that the greatest improvements over the baseline are found the lower the R0 value is.}
\end{figure}
\newpage


\section{Discussion}

\subsection{Analysis of total infection size}

In this section, we analyse the total percentage of agents that became infected in a run of the disease. Each of the models was run over 4 different R0 virus base reproduction rates. The higher the Disease R0 the more infectious it is. Inspection of figure 11 shows the base 50\% compliance baseline saw an improvement from -5\% at the high R0 values to 17\% at the low R0 values. The non-strategic model was found to be the most effective in reducing the number of infected agents throughout the R0 values of 2 (16\%), 3 (6\%) and 5.4 (0\%) but by 9.5, the strategic community size model was found to be more effective with a 4\% reduction of disease spread (figure 11). This could be a result of the non-strategic model taking into account both the global spread of the disease in the network as well as the current compartment / state of close contacts impacting compliance. In the case of an extremely high R0 values the strategic community size model was most effective.  This model only considers the number of close contacts an agent has when deciding if they are compliant or not. This could be the case as it forces those most central agents to be compliant. Therefore, they are not able to spread the disease far across the network, forcing it to stay more localised where it has the opportunity to die out before it becomes an epidemic in the network.\newline 

The models were likely more effective on the lower R0 values as the agents have more time to choose to be compliant. As the disease spreads slower though the system, there is more time for compliant agents to go through a run of surveillance testing and potentially isolate before they have infected many other individuals. With extremely high R0 numbers it is observed that the non-strategic and strategic behavioural model have little effect on the total number infected (i.e. all having $< 5$\% improvement). \newline 

The models which allow agents compliance to change had the most improvement when the R0 was 2. Specifically as seen in figure 6, the baseline had a median of 16.5\% infected and Strategic Local State had a median of 9\%infected. In comparison the model which used a baseline of 100\% saw the greatest difference with an R0 of 3. Figure 7 shows the 50\% baseline had a median of 73\% infected while with the 100\% baseline it was 13.5\% as seen in figure 10. This suggests that with a R0 of 3 and greater, agents do not have enough time to choose to be compliant and see the maximum benefits of surveillance and symptomatic case testing. However of the methods of voluntary compliance with the R0 of 3, the best at reducing infected agents was the non strategic minimum 50\%. This suggests with moderate disease spread levels, agents that rely on a combination of the local and global situation are better at lowering the spread of the disease in the network. \newline
\newpage

\subsection{Analysis of Epidemic Duration}
In this section, we analysse the number of days the disease took to stop spreading in the network when no agents are infectious. We found that the strategic models particularly at the lower R0 values of 2 and 3 saw significant improvement over the 50\% baseline tests, Regarding the days the simulation took before disease spread stopped. This is interesting as while similar percentages of agents were infected, the time it took for the disease to die out was less, suggesting the disease spread through the network faster but ultimately had about the same proportion of agents who ended up getting the disease.\newline 

Notably the case of Strategic Local State where agents use the number of agents around them who are in a high risk state to decide if they are compliant fared the worst at a disease R0 value of 9.5 as can be seen in figure 11, having a 2\% increase in infected agents and it taking 6\% longer for the disease to stop spreading. This would suggest that agents who based decisions about if to test based solely on how their close contacts are faring is not a good approach at reducing the spread of disease and more global measures or measures that rely on how central a person is in their network of close contacts are more effective. \newline 


\subsection{Future Work and Conclusion}
In this section we disucss potential areas of future work and conclude this paper. There are a few areas in which future work could be conducted, some areas would include; Modifying the model to allow for vaccinated agents, as this would more accurately mirror the real world situation we now have with a vaccinated population. Creating a network which assigns the agents ages which would have an impact on various parts of the model such as disease test false negative rates, hospitalisation and fatality rates as well as potentially R0 values for the specific agents. This would create a more accurate simulation as the false negative rate for rapid tests varies significantly with age and for the purposes of this paper the disease and testing parameters for infants / young children were applied to everyone. Furthermore, the models could be expanded by forcing all or sections of the network to enter quarantine for a period of time under certain conditions, to control spread in the network. This would be able to simulate the day-care closing for a period of time due to a very high spread of the disease. \newline 

Overall, the faster the disease spreads though the network the less effective non pharmaceutical interventions such as surveillance testing is. Moreover, models which allow agents’ compliance to change over time experience this to a greater effect. This suggests relying on non pharmaceutical  interventions alone is not very effective when a disease spreads rapidly. Other non pharmaceutical  interventions which do not rely on voluntary compliance may have to be relied on. For the childcare setting of this network it may involve closing off an entire cohort of the network preventing agents in that cohort from visiting the childcare and potentially spreading the disease to others. Interventions such as this could be explored in future work and its effectiveness compared to purely voluntary non pharmaceutical interventions.

\newpage

\bibliography{bibliography}{}
\bibliographystyle{plain}

\newpage
\appendix

\section{Appendix}
The code for the models can be found at \url{https://github.com/robertmxmx/seirsplus-dynamic-agents}.
The format for this paper is based on a journal submission for PLOS Computational Biology.

\end{document}


\end{document}

